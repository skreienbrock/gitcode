\documentclass[]{scrartcl}

%opening
\title{using docker:\\a docker cheat sheet}
\author{Sven Kreienbrock}
%packages
\usepackage{tcolorbox}

\begin{document}

% new boxes
\newtcolorbox{mybox}
{
    colback=black!5!white,
	fonttitle=\fontfamily{courier},
	colframe=black!50!white,
}
\maketitle

\begin{abstract}
% some info
\end{abstract}

\section{Getting Information}

\begin{mybox}
	\begin{itemize}
		\item Collect information about docker itself (\textit{the docker daemon \textsf{dockerd}})
		\subitem \texttt{\$ docker info} 
		\subitem \texttt{\$ docker version}
		\item How much storage does docker use on your local hard disk?
		\subitem  \texttt{\$ docker system df}
		\item Deleting all stopped containers, all networks, the build cache and reclaim system space
		\subitem \texttt{\$ docker system prune}
		\item Getting infos about downloaded images
		\subitem \texttt{\$ docker images}
		\item Manage active [or inactive] Containers
		\subitem \texttt{\$ docker ps [-a]}
		\item Check on Ressources (Container Usage)
		\subitem \texttt{\$ docker stats}
		\item Searching for docker images (CLI based)
		\subitem \texttt{\$ docker search keras} or \texttt{\$ docker search tensorflow} e.g.
		\subitem \(\rightarrow\) It is necessary to check Docker-Hub for Usage (Invocation)
		\subitem \(\rightarrow\) \texttt{https://hub.docker.com}
	\end{itemize}
\end{mybox}

\section{executing/running docker images}
\begin{mybox}
	Some basics you should consider when executing images: \\
	The first step always: \texttt{docker run [OPTIONS]}, where OPTIONS could be one or more of:
	\begin{itemize}
		\item \texttt{-v; --volume list} adding Support for a Volume 
		\subitem e.g.: \texttt{-v /path/at/host:/path/in/container}
		\item \texttt{-p; --publish list} publishing of container internal ports
		\subitem e.g.: \texttt{-p 8888:8888} (mapping host-Port 8888 to container Port 8888)
		\item \texttt{-i} interactive (mostly in combination with \texttt{-t} for tty)
		\item \texttt{-t} tty (mostly in combination with \texttt{-i} for interactive tty)
		\item \texttt{--rm} removes container after usage
		\item \texttt{-e} setting of environment variables
		\subitem e.g.: \texttt{-e DISPLAY=\$DISPLAY}  
		\subitem sets Variable DISPLAY to systems \$DISPLAY. 
	\end{itemize}
\end{mybox}

\section{Examples}
\noindent
Example: run the BPH Matlab Image, with X11 forwarding
\begin{mybox}
	\texttt{\$ xhost +; docker run -it \\
		-v /tmp/.X11-unix:/tmp/.X11-unix \\
		-v /bph/home/user:/data \\
		-e DISPLAY=\$DISPLAY \\
		svekre/ml:latest matlab; xhost -}
\end{mybox}
\\
\noindent
Example: R-Studio (login with rstudio and \texttt{MYPASSWORDX})
\begin{mybox}
	\begin{itemize}
		\item 	\texttt{\$ docker run -d -p 8787:8787 \\
			-e PASSWORD=MYPASSWORDX \\
			--name rstudio rocker/rstudio}
	    \subitem \(\rightarrow\) Afterwards type into your browser address bar: 
	    \subitem  \(\rightarrow\) \texttt{http://localhost:8787} and provide your login credentials.
	\end{itemize}
\end{mybox}

\section{stop and delete images/containers}
\begin{mybox}
	Example: container named cringy\_conti as Instance of Image rocker/rstudio
	\begin{itemize}
		\item \texttt{\$ docker rm cringy\_conti} 
		\subitem deletes container
		\item \texttt{\$ docker rmi rocker/rstudio}
		\subitem removes image from Host
	\end{itemize}
\end{mybox}

\end{document}
